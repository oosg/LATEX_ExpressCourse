% Ejemplo 20. ECUACIONES

\documentclass{article}

\usepackage{amsmath}  
\usepackage{mathtools}  
\usepackage{xfrac} 

\title{EJEMPLOS \LaTeX DE ECUACIONES}

\begin{document}
\maketitle
    \begin{equation}  
        x + y = 4 % there should be no gap between any of the two rows  
    \end{equation}  
        \newline
        
    \begin{equation}      
        x^2 +2x+ 6 = 0               
    \end{equation}  
    \begin{equation} 
        x^3 + y^3 = 9   
    \end{equation}  
    \begin{equation}  
        x^2 + 2x + 4 = 0   
    \end{equation}  
    \begin{equation}  
        y^2 + 4y = 5   
    \end{equation} 
    \newline

    \begin{equation}  
        \cos^2 \theta + \sin^2 \theta = 1  
    \end{equation}  
    \newline

    %POWER AND INDICES
    \begin{equation}
        k_n = k_{n-1} + k_{n-2}  
    \end{equation}

    \begin{align*}  
        r(x) = \frac{1}{x^2 + y^2 + z^2} = 1 \\ \\ % you can insert any type of fraction according to the requirements.  
        g(x) = \frac{\frac{1}{a}+\frac{1}{b}}{a - b} \\ \\  
        f(x) = \frac{a - b}{\frac{1}{a} - \frac{1}{b}} \\  
    \end{align*}  

    \begin{equation}  
        \sum_{i=0}^{N - 1} a_i % you can modify the values according to the requirements  
    \end{equation}  


    \[  
        \begin{bmatrix} % you can specify any environment according to your choice  
        A & B & C \\  
        D & E & F \\  
        G & H & I   
        \end{bmatrix}  
        =  
        \begin{bmatrix}  
        J & K & L \\  
        M & N & O \\  
        P & Q & R  
        \end{bmatrix}  
    \]  

    \[ f(x) =  
        \begin{cases}  
            x^2 + 2x       & \quad \text{if } x \text{ is greater than 0}\\ % the text command is just used for the formatting  
            0  & \quad \text{if } x \text{ is less than 0} % the \quad command maintains the distance between the text and the math variable  
        \end{cases}  
    \]  

    \[  
         \left( \int_{0}^{n} f(x)\,dx\right)=g(x).  
    \]  

    \[  
      \frac{d}{dy}\left( \int_{0}^{y} f(x)\,dx\right)=g(y).  
    \]  


    \begin{equation} 
    x= \frac{-b \pm \sqrt{b^2 -4ac}}{2a}
    \end{equation}

    This sentence uses  delimiters to show math inline:  $\sqrt{3x-1}+(1+x)^2$

    
\end{document}
    


