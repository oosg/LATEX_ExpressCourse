
%EJEMPLO 17.  TABLES

\documentclass{article}
\usepackage{rotating}

\title{EJEMPLOS CON TABLAS}
\author{Autor: Oscar Sandoval}



\begin{document}
\maketitle
  \section{A simple Table}
    \begin{table}[!hbt]
        \centering
        \caption{Obtained marks.}
        \label{tab-marks}
        \begin{tabular}{|c|c|c|c|c|}  %Genera 5 columnas
            \hline Name & Math & Phy & Chem & English\\ %el & separa los valores
            \hline Robin & 80 & 68 & 60 & 57\\
            \hline Julie & 72 & 62 & 66 & 63\\
            \hline Robert & 75 & 70 & 71 & 69\\
            \hline
        \end{tabular}
    \end{table}

\section{Tabla con entradas de dirección vertical}
  \begin{table}[!hbt]
        \centering
        \caption{Tabla con entradas de dirección vertical}
        \label{tab-marks_vertical}
    \begin{tabular}{|c|c|c|c|}
        \hline Name& \begin{sideways}Mathematics\end{sideways}&\begin{sideways}Physics\end{sideways}&\begin{sideways}Chemistry\end{sideways}\\
        \hline Robin & 80 & 68 & 60\\
        \hline Julie & 72 & 62 & 66\\
        \hline Robert & 75 & 70 & 71\\
        \hline
    \end{tabular}
 \end{table}

Como se puede apreciar en la Tabla \ref{tab-marks} es una tabla simple. Y en la tabla \ref{tab-marks_vertical} es una tabla con entradas en dirección vertical.

\end{document}

If ! is used, many default or preset restrictions are ignored and a table is attempted
to put in the specified position.

If h is given, the table is attempted to put in the exact position. If fails and no
more specifier is given, by default LATEX considers the specifier t for placing the
table on the top of the next page.

If t is given, the table is attempted to put on the top of the current page.

If b is given, the table is attempted to put at the bottom of the current page.