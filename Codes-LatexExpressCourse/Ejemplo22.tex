% ECUACIONES 

\documentclass{article}

\usepackage{amsmath}
\usepackage{mathtools}
\usepackage{xfrac}

\title{EJEMPLOS DE ECUACIONES}

\begin{document}
\maketitle
 
como dice la ecuación \ref{eq:1}

     \begin{equation}
     \label{eq:1}
        x+y=4;   
     \end{equation}
     
     \begin{equation}
     \label{eq:2}
        x^2+2x+6=0 
     \end{equation}

     \begin{equation}
         x^3+y^3=9
     \end{equation}

     \begin{equation}
         \cos^2\theta+\sin^2\theta=1
     \end{equation}

     \begin{equation}
         k_n=k_{n-1}+k_{n-2}
     \end{equation}

     \begin{align*}
         r(x)=\frac{1}{x^2+y^2+z^2}=1  \\
         g(x)=\frac{\frac{1}{a}+\frac{1}{b}}{a-b} \\
         f(x)=\frac{a-b}{\frac{1}{a}-\frac{1}{b}}
     \end{align*}

     \begin{equation}
         \sum_{i=0}^{N-1}a_i
     \end{equation}



\begin{equation}
    \begin{bmatrix}
        A&B&C\\
        D&E&F\\
        G&H&I
    \end{bmatrix}
=
    \begin{bmatrix}
        J&K&L\\
        M&N&O\\
        P&Q&R
    \end{bmatrix}
\end{equation}

\begin{equation}
    f(x)=
    \begin{cases}
        x^2+2x   &\quad \text{if } x \text{ is greater than 0}\\
        0        &\quad \text{is less than 0}
    \end{cases}
\end{equation}

\begin{equation}
    \left(\int_{0}^{n}f(x)\,dx \right)=g(x)
\end{equation}

\begin{equation}
   \frac{d}{dy}\left(\int_{0}^{y}f(x)\,dx\right)=g(y)
\end{equation}

This sentece uses a delimiters to show math inline: $\sqrt{3x-1}+(1+x)^2$
     
\end{document}